% Dokumententyp, benutzte Pakete und deren Einstellungen
\documentclass[	\hsmasprache,%
				\hsmaabgabe,%
				\hsmapublizieren,%
				\genderhinweis,
				\hsmaquellcode,
				\hsmasymbole,
				\hsmaglossar,
				\hsmatc]{HMA}

% Wo sind die Bilder?
\graphicspath{{bilder/}}


% Wo liegt Sourcecode?
\newcommand{\srcloc}{src/}


% Checklisten mit zwei Ebenen
\newlist{checklist}{itemize}{2}
\setlist[checklist]{label=$\square$}



% Befehl zum Erstellen eigener Makros. In diesem Fall ist es ein Makro für das Einbinden von Bildern. Das label (für \ref) ist dann der Name der Bilddatei
\newcommand{\bild}[3]{
	\begin{figure}[ht]
		\centering
		\includegraphics[width=#2]{#1}
		\caption{#3}
		\label{#1}
\end{figure}}




							


\newcommand{\snowcard}[9]{
	\begin{table}[ht!]
		\caption{\hsmasnowcardanforderung\ #1 -- #4}\label{#1}
		\renewcommand{\arraystretch}{1.2}
		\centering
		\sffamily
		\begin{footnotesize}

			\begin{tabularx}{\linewidth}{sssssl}
				\toprule
				\textbf{\hsmasnowcardno} & #1 & \textbf{\hsmasnowcardart} & #2 & \textbf{\hsmasnowcardprio} & #3 \\
				\midrule
				\multicolumn{2}{l}{\textbf{\hsmasnowcardtitel}} & \multicolumn{4}{l}{\parbox[t]{11.8cm}{#4}} \\
				\ifx&#5&%
				\else
				\multicolumn{2}{l}{\textbf{\hsmasnowcardherkunft}} & \multicolumn{4}{l}{\parbox[t]{11.8cm}{#5}} \\
				\fi
				\ifx&#6&%
				\else
				\multicolumn{2}{l}{\textbf{\hsmasnowcardkonflikt}} & \multicolumn{4}{l}{\parbox[t]{11.8cm}{#6}} \\
				\fi
				\addlinespace
				\multicolumn{6}{l}{\textbf{\hsmasnowcardbeschreibung}} \\
				\multicolumn{6}{l}{\parbox[t]{13.5cm}{#7\strut}} \\
				\ifx&#8&%
				\else
				\addlinespace
				\multicolumn{6}{l}{\textbf{\hsmasnowcardfitkriterium}} \\
				\multicolumn{6}{l}{\parbox[t]{13.5cm}{#8\strut}} \\
				\fi
				\ifx&#9&%

				\else
				\addlinespace
				\multicolumn{6}{l}{\textbf{\hsmasnowcardmaterial}} \\
				\multicolumn{6}{l}{\parbox[t]{13.5cm}{#9\strut}} \\
				\fi
				\bottomrule
			\end{tabularx}
		\end{footnotesize}
	\end{table}
}



% Quality Attribute Scenario
\newcommand{\qas}[9]{
	\begin{table}[ht!]
		\caption{\hsmaqasanforderung\ #1 -- #3}\label{#1}
		\renewcommand{\arraystretch}{1.2}
		\centering
		\sffamily
		\begin{footnotesize}

			\begin{tabularx}{\linewidth}{sssssl}
				\toprule
				\textbf{\hsmaqasno} & #1 & \textbf{\hsmaqasart} & QAS & \textbf{\hsmaqasprio} & #2 \\
				\midrule
				\multicolumn{2}{l}{\textbf{\hsmaqastitel}} & \multicolumn{4}{l}{\parbox[t]{11.8cm}{#3}} \\
				\multicolumn{2}{l}{\textbf{\hsmaqasquelle}} & \multicolumn{4}{l}{\parbox[t]{11.8cm}{#4}} \\
				\multicolumn{2}{l}{\textbf{\hsmaqasstimulus}} & \multicolumn{4}{l}{\parbox[t]{11.8cm}{#5}} \\
				\multicolumn{2}{l}{\textbf{\hsmaqasartefakt}} & \multicolumn{4}{l}{\parbox[t]{11.8cm}{#6}} \\
				\addlinespace
				\multicolumn{6}{l}{\textbf{\hsmaqasumgebung}} \\
				\multicolumn{6}{l}{\parbox[t]{13.5cm}{#7\strut}} \\
				\addlinespace
				\multicolumn{6}{l}{\textbf{\hsmaqasantwort}} \\
				\multicolumn{6}{l}{\parbox[t]{13.5cm}{#8\strut}} \\
				\addlinespace
				\multicolumn{6}{l}{\textbf{\hsmaqasmass}} \\
				\multicolumn{6}{l}{\parbox[t]{13.5cm}{#9\strut}} \\
				\bottomrule
			\end{tabularx}
		\end{footnotesize}
	\end{table}
}

