%% !TeX program = lualatex

%\listfiles	
% **************************
% STOP - Bitte zuerst lesen, bevor Sie weitermachen
%
% Einige Dinge müssen Sie an Ihre Bedürfnisse (und die Vorgaben Ihres
% Betreuers anpassen. Editieren Sie dazu die Datei docinfo.tex).
%
% 1. Sprache
% Das Template unterstützt Deutsch und Englisch, Standard ist Deutsch.
% Wenn Sie Englisch verwenden wollen, ändern Sie bitte direkt am Anfang
% dieser Datei den Eintrag
%    \newcommand{\hsmasprache}{de}
% auf
%    \newcommand{\hsmasprache}{en}
%
% 2. Form der Abgabe
% Das Template unterstützt sowohl eine digitale Abgabe, als auch eine Abgabe
% auf Papier. Bei einer Papierabgabe wird ein doppelseitiger Druck vorbereitet
% und der Titel wird so platziert, dass er in das Fenster des offiziellen
% Umschlages der Technischen Hochschule passt.
% Bei einer digitalen Abgabe (als PDF) wird der Titel zentriert und als
% Format wird einseitig gewählt. Außerdem wird die Datei unterschrift.png
% auf dem Blatt mit der Erklärung zur Eigenständigkeit eingebunden.
%
% 3. Zitierstil
% Abhängig von dem gewünschten Zitierstil passen Sie bitte in
% hma.cls die Einstellungen bei \RequirePackage[backend=biber...
% an. Wie ist dort genau erklärt.
% Achtung: Wenn Sie als Zitierstil Fußnoten wählen bzw. generell
% -------  mit Fußnoten arbeiten, dann beachten Sie bitte, dass
%          Fußnoten in Bildunterschriften und Tabellenüberschriften
%          nicht funktionieren.
%          Siehe hierzu https://texfaq.org/FAQ-ftncapt
%          und https://texfaq.org/FAQ-footintab
%          Sinnvollerweise verzichten Sie auf Fußnoten an diesen
%          Stellen und fügen Quellen einfach per \parancite ein.
%
% 4. Doppelseitiger oder einseitiger Druck
% Das Template bestimmt, ob einseitig oder doppelseitig gedruckt wird
% anhand der Abgabeform (papier / digital). Wollen sie dies übersteuern,
% müssen Sie in der Datei preambel.tex folgende Zeile
% \KOMAoptions{twoside=true} für doppelsetigen Druck
% \KOMAoptions{twoside=false} für einsetigen Druck
% direkt vor \usepackage{xcolor} einsetzen. Prinzipiell sollten Sie aber
% das vorgeschlagene Format einfach so lassen.
%
% 5. Unnötige Teile entfernen
% Entfernen Sie die Teile, die Sie nicht brauchen, z.B. Anhänge
%
% 6. Silbentrennung
% LaTeX führt eine automatische Silbentrennung durch. Allerdings
% werden Wörter, die bereits einen Bindestrich enthalten nicht
% getrennt, z.B. Datenschutz-Grundverordnung. Wenn Sie Ihren Text auf
% Deutsch schreiben, können Sie dann alternativ "= für den Bindestrich
% im Wort verwenden, z.B. Datenschutz"=Grundverordnung, damit LaTeX
% weiterhin richtig trennt.
% Ist die Silbentrennung aus einem anderen Grund nicht erfolgt, sodass
% das Wort über den rechten Rand hinaussteht oder wenn Sie eine weitere
% Trennstelle wollen, können Sie LaTeX helfen, indem Sie weitere
% Trennstellen angeben. Dies geschieht durch "- als Zeichen, z.B.
% Staats"-vertrag.
%
% 7. Nummerierung der Fußnoten
% LaTeX beginnt die Nummerierung der Fußnoten in jedem Kapitel wieder
% bei 1. In diesem Template wird die fortlaufende Nummerierung der Fußnoten 
% über die gesamte Arbeit umgesetzt.  
%
% 8. Unterschrift
% Bei einer Abgabe auf Papier unterschreiben Sie die Arbeit eigenhändig.
% Geben Sie allerdings digital ab, sollten Sie die Datei unterschrift.png in dem 
% Ordner /bilder durch einen Scan Ihrer eigenen Unteschrift ersetzen - andernfalls
% unterschreiben Sie als Max Mustermann ;-)
% *******************************************************************


% Dokumenteneinstellungen und Infos importieren
% -------------------------------------------------------
% Informationen und Einstellungen für Ihre Abschlussarbeit
%

% Sprache für das Dokument festlegen
\newcommand{\hsmasprache}{de} %de für Deutsch oder en für Englisch


% Abgabeform festlegen
% Bei einer digitalen Abgabe, wird das Dokument einseitig erzeugt und der Titel wird
% zentriert.
\newcommand{\hsmaabgabe}{digital} % Abgabe erfolgt für Fakultät I digital. Optionen hier sind für anderen Fakultäten: "papier" oder "digital".


% Flags für Veröffentlichung, Sperrvermerk
\newcommand{\hsmapublizieren}{opensource}   
% Wird einer Veröffentlichung zugestimmt?
% Optionen: 
% opensource = Druck der CC Lizenz mit By SA (Standard)
% hs = Veröffentlichung an der Technischen Hochschule und auf Hochschulservern
% stud = kein opensource und keine veröffentlichung auf den Hochschulservern
% vertraulich = Arbeit darf nicht veröffentlicht werden und erhält einen Sperrvermerk (Nur nach Absprache mit Betreuer setzen!)


\newcommand{\genderhinweis}{gender}     % Soll der Gender-Hinweis angezeigt werden? ja=gender, nein = nogender; Genderhinweis wird nur in deutscher Sprache angezeigt!


\newcommand{\hsmaquellcode}{sourcecode} % Verwenden Sie Quellcode in Ihrer Arbeit? ja=sourcecode, nein= nosourcecode

\newcommand{\hsmasymbole}{symbole} % Verwenden Sie viele Symbole in Ihrer Arbeit, welche in einem Symbolverzeichnis aufgeführt werden sollen? ja=symbole, nein= nosymbole


\newcommand{\hsmaglossar}{glossar} % Verwenden Sie Begriffserklärungen nicht Abkürzungen in Ihrer Arbeit? ja=glossar, nein= noglossar

\newcommand{\hsmatc}{tc} % Verwenden der Änderungsmarkierung. Änderungsmarkierung aktiv und eine Liste der Änderungen wird angezeigt = tc, Keine Änderungsmarkierung und keine Ausgabe der Änderungen = notc




% Titel der Arbeit auf Deutsch
\newcommand{\hsmatitelde}{Einsatz eines Flux-Kompensators für Zeitreisen mit einer maximalen Höchstgeschwindigkeit von WARP~7}

% Titel der Arbeit auf Englisch
\newcommand{\hsmatitelen}{Application of a flux compensator for timetravel with a maximum velocity of warp~7}

% Weitere Informationen zur Arbeit
\newcommand{\hsmaort}{Mannheim}          % Ort
\newcommand{\hsmaautorvname}{Max}        % Vorname(n)
\newcommand{\hsmaautornname}{Mustermann} % Nachname(n)
\newcommand{\hsmaabgabedatum}{2025-03-04}% Datum der Abgabe in dem Format JJJJ-MM-TT

\newcommand{\hsmafirma}{Paukenschlag GmbH, Mannheim} % Firma bei der die Arbeit durchgeführt wurde
\newcommand{\hsmabetreuer}{Prof. Peter Mustermann, Technische Hochschule Mannheim} % Betreuer an der Hochschule
\newcommand{\hsmazweitkorrektor}{Erika Mustermann, Paukenschlag GmbH}   % Betreuer im Unternehmen oder Zweitkorrektor

\newcommand{\hsmafakultaet}{I}    % I für Informatik oder E, S, B, D, M, N, W, V
\newcommand{\hsmastudiengang}{IB} % IB IMB UIB CSB IM MTB (weitere siehe titleblatt.tex)


% Preambel importieren
% Document type and used packages
\documentclass[open=right, % Kapitel darf nur auf rechten Seite beginnen
paper=A4,               % DIN-A4-Papier
a4paper,                % DIN-A4-Papier
12pt,                   % Schriftgöße
headings=small,         % Kleine Überschriften
headsepline=true,       % Trennlinie am Kopf der Seite
footsepline=false,      % Trennlinie am Fuß der Seite
bibliography=totoc,     % Literaturverzeichnis in das Inhaltsverzeichnis aufnehmen
twoside=on,             % Doppelseitiger Druck - auf off stellen für einseitig
DIV=7,
cleardoublepage=plain]{scrbook} 

% Pakete einbinden, die benötigt werden
\usepackage[utf8]{inputenc}       % Dateien in UTF-8 benutzen
\usepackage[T1]{fontenc}          % Zeichenkodierung
\usepackage{graphicx}             % Bilder einbinden
\usepackage[ngerman,english]{babel}       % Deutsch und Englisch unterstützen
\usepackage{xcolor}               % Color support
\usepackage{amsmath}              % Matheamtische Formeln
\usepackage{amsfonts}             % Mathematische Zeichensätze
\usepackage{amssymb}              % Mathematische Symbole
\usepackage{float}                % Fließende Objekte (Tabellen, Grafiken etc.)
\usepackage{booktabs}             % Korrekter Tabellensatz
\usepackage[printonlyused]{acronym}   % Abkürzungsverzeichnis [nur verwendete Abkürzugen]
\usepackage{makeidx}              % Sachregister
\usepackage{fancyhdr}             % Schönere Überschriften
\usepackage{listings}             % Source Code listings
\usepackage{listingsutf8}         % Listings in UTF8
\usepackage[hang,font={sf,footnotesize},labelfont={footnotesize,bf}]{caption} % Beschriftungen
\usepackage[scaled]{helvet}       % Schrift Helvetia laden
\usepackage[sf,bf,small]{titlesec} % Einstellungen für Überschriften
\usepackage[absolute]{textpos}	% Absolute Textpositionen (für Deckblatt)
\usepackage{calc}                 % Berechnung von Positionen
\usepackage{blindtext}            % Blindtexte
\usepackage[bottom=40mm,left=35mm,right=35mm,top=30mm]{geometry} % Ränder ändern
\usepackage[square]{natbib}       % Literaturverzeichnis nach DIN mit eckigen Klammern bei \citep
\usepackage{setspace}             % Abstände korrigieren
\usepackage{ifthen}               % Logische Bedingungen mit ifthenelse
\usepackage{scrhack}              % Get rid of tocbasic warnings
\usepackage[pagebackref=false]{hyperref}  % Hyperlinks
\usepackage[all]{hypcap}          % Korrekte Verlinkung von Floats

% Farben definieren
\definecolor{linkblue}{RGB}{0, 0, 100}
\definecolor{linkblack}{RGB}{0, 0, 0}
\definecolor{comment}{RGB}{63, 127, 95}
\definecolor{darkgreen}{RGB}{14, 144, 102}
\definecolor{darkblue}{RGB}{0,0,168}
\definecolor{darkred}{RGB}{128,0,0}
\definecolor{javadoccomment}{RGB}{0,0,240}

% Einstellungen für das Hyperlink-Paket
\hypersetup{
    colorlinks=true,      % Farbige links verwenden       
%    allcolors=linkblue,
    linktoc=all,          % Links im Inhaltsverzeichnis
    linkcolor=linkblack,  % Querverweise
    citecolor=linkblack,  % Literaturangaben
	filecolor=linkblack,  % Dateilinks
	urlcolor	=linkblack    % URLs
}

% Einstellungen für Quelltexte
\lstset{     
      xleftmargin=0.2cm,     
      basicstyle=\footnotesize\ttfamily,
      keywordstyle=\color{darkgreen},
      identifierstyle=\color{darkblue},
      commentstyle=\color{comment}, 
      stringstyle=\color{darkred}, 
      tabsize=2,
      lineskip={2pt},
      columns=flexible,
      inputencoding=utf8,
      captionpos=b,
      breakautoindent=true,
	  breakindent=2em,
	  breaklines=true,
	  prebreak=,
	  postbreak=,
      numbers=none,
      numberstyle=\tiny,
      showspaces=false,      % Keine Leerzeichensymbole
      showtabs=false,        % Keine Tabsymbole
      showstringspaces=false,% Leerzeichen in Strings
      morecomment=[s][\color{javadoccomment}]{/**}{*/},
      literate={Ö}{{\"O}}1 {Ä}{{\"A}}1 {Ü}{{\"U}}1 {ß}{{\ss}}2 {ü}{{\"u}}1 {ä}{{\"a}}1 {ö}{{\"o}}1
}

\urlstyle{same}

% Einstellungen für Überschriften
\titlespacing{\paragraph}{0pt}{1ex}{2.0ex}
\titlespacing{\subsubsection}{0pt}{3ex}{0.0ex}
\titlespacing{\subsection}{0pt}{4ex}{0.2ex}
\titlespacing{\section}{0pt}{7ex}{1ex}
\titleformat*{\subsubsection}{\sffamily\itshape\bfseries\small}
\titleformat*{\paragraph}{\sffamily\bfseries\small}

% Einstellungen für Schriftarten
\setkomafont{pagehead}{\normalfont\sffamily}
\setkomafont{pagenumber}{\normalfont\sffamily}
\addtokomafont{footnote}{\footnotesize}

% Wichtige Abstände
\setlength{\parskip}{0.2cm}  % 2mm Abstand zwischen zwei Absätzen
\setlength{\parindent}{0mm}  % Absätze nicht einziehen
\clubpenalty = 10000         % Keine "Schusterjungen"
\widowpenalty = 10000        % Keine "Hurenkinder"
\displaywidowpenalty = 10000 % Keine "Hurenkinder"
\renewcommand{\footnotesize}{\fontsize{9}{10}\selectfont} % Größe der Fußnoten
\setlength{\footnotesep}{8pt} % Abstand zwischen den Fußnoten

% Index erzeugen
\makeindex

% Einfacher Font-Wechsel über dieses Makro
\newcommand{\changefont}[3]{
\fontfamily{#1} \fontseries{#2} \fontshape{#3} \selectfont}

% Eigenes Makro für Bilder
\newcommand{\bild}[3]{
\begin{figure}[h]
  \centering
  \includegraphics[width=#2]{#1}
  \caption{#3}
  \label{#1}
\end{figure}}

% Wo liegt Sourcecode?
\newcommand{\srcloc}{src/}

% Wo sind die Bilder?
\graphicspath{{bilder/}}

% Makros für typographisch korrekte Abkürzungen
\newcommand{\zb}[0]{z.\,B.\ }
\newcommand{\dahe}[0]{d.\,h.\ }
\newcommand{\ua}[0]{u.\,a.\ }



%Abstrakt Ihrer Abschlussarbeit
% -------------------------------------------------------
% Abstrakt / Abstract
% Achtung: Wenn Sie im Abstrakt Anführungszeichen verwenden wollen, dann benutzen Sie
%          nicht "` und "', sondern \enquote{}. "` und "' werden nicht richtig
%          erkannt.

% Kurze (maximal halbseitige) Beschreibung, worum es in der Arbeit geht auf Deutsch
\newcommand{\hsmaabstractde}{Jemand musste Josef K. verleumdet haben, denn ohne dass er etwas Böses getan hätte, wurde er eines Morgens verhaftet. Wie ein Hund! sagte er, es war, als sollte die Scham ihn überleben. Als Gregor Samsa eines Morgens aus unruhigen Träumen erwachte, fand er sich in seinem Bett zu einem ungeheueren Ungeziefer verwandelt. Und es war ihnen wie eine Bestätigung ihrer neuen Träume und guten Absichten, als am Ziele ihrer Fahrt die Tochter als erste sich erhob und ihren jungen Körper dehnte. Es ist ein eigentümlicher Apparat, sagte der Offizier zu dem Forschungsreisenden und überblickte mit einem gewissermaßen bewundernden Blick den ihm doch wohl bekannten Apparat. Sie hätten noch ins Boot springen können, aber der Reisende hob ein schweres, geknotetes Tau vom Boden, drohte ihnen damit und hielt sie dadurch von dem Sprunge ab. In den letzten Jahrzehnten ist das Interesse an Künstlern sehr zurückgegangen. Aber sie überwanden sich, umdrängten den Käfig und wollten sich gar nicht fortrühren.}

% Kurze (maximal halbseitige) Beschreibung, worum es in der Arbeit geht auf Englisch
\newcommand{\hsmaabstracten}{The European languages are members of the same family. Their separate existence is a myth. For science, music, sport, etc, Europe uses the same vocabulary. The languages only differ in their grammar, their pronunciation and their most common words. Everyone realizes why a new common language would be desirable: one could refuse to pay expensive translators. To achieve this, it would be necessary to have uniform grammar, pronunciation and more common words. If several languages coalesce, the grammar of the resulting language is more simple and regular than that of the individual languages. The new common language will be more simple and regular than the existing European languages. It will be as simple as Occidental; in fact, it will be Occidental. To an English person, it will seem like simplified English, as a skeptical Cambridge friend of mine told me what Occidental is.}



% Literatur-Datenbank
\addbibresource{literatur.bib}   % BibLaTeX-Datei mit Literaturquellen einbinden


% Anfang des Dokuments
\begin{document}	


%Laden der Dateien mit Abkürzungen, Begriffserklärungen (Glossar), Symbole
\chapter*{Abkürzungsverzeichnis}
\addcontentsline{toc}{chapter}{Abkürzungsverzeichnis}

\begin{acronym}
\acro{ABK}{Abkürzung}
\acro{ACM}{Association of Computing Machinery}
\acro{PDF}{Portable Document Format}
\acro{IEEE}{Institute of Electrical and Electronics Engineers}
\acro{ISO}{International Organization for Standardization}
\end{acronym}

% Glossareinträge
\newglossaryentry{glos:amplification}{name={Amplification}, description={describes the disproportionate increase of a response packet compart to the initial request packet.}}

% Verzeichnis von Symbolen und Einheiten
\newglossaryentry{symb:Pi}{name=\ensuremath{\pi},
	description={Geometrical value},
	unit={},
	type=symbolslist}

\newglossaryentry{symb:energyconsump}{name=\ensuremath{P},
	description={Energy consumption},
	unit={\si{kW}},
	type=symbolslist}
	
	
\pagestyle{headings}
\tableofcontents


\mainmatter

%% Ihre Inhaltsdateien werden an dieser Stelle in das Dokument eingefügt
\chapter{Einleitung}

\section{Erster Abschnitt}

Einleitung zur Arbeit.

Möglicherweise noch einmal unterteilt in Unterabschnitte.


\subsection{Textauszeichnungen}
\label{Einleitung:Textauszeichnungen}
\index{Auszeichnungen!im Text}

Man kann Text auch \textit{kursiv} oder \textbf{fett} setzen. Es gibt Bindestrichte -, Gedankenstriche -- und lange Striche ---.


\subsection{Anführungszeichen}

Deutsche Anführungszeichen gehen so: "`dieser Text steht in \glq Anführungszeichen\grq; alles klar?"'.


\subsection{Abkürzungen}
\index{Abkürzungen}
\index{Abbreviation|see{Abkürzungen}}

Eine \ac{AKÜ} wird bei der ersten Verwendung ausgeschrieben\footnote{Ausschreiben bedeutet, dass man nicht die Abkürzung sondern die lange Form verwendet.}. Danach nicht mehr: \ac{AKÜ}. Man kann allerdings die Langform\footnote{\blindtext} explizit anfordern: \acl{AKÜ} oder die Kurzform \acs{AKÜ} oder auch noch einmal die Definition: \acf{AKÜ}.

Mehr dazu findet sich im Kapitel~\ref{Einleitung:Textauszeichnungen} auf Seite~\pageref{Einleitung:Textauszeichnungen}.




\subsubsection{Noch ein Unterabschnitt}

\paragraph{Eine Absatzüberschrift}

\blindtext[1]


\subsection{Literaturarbeit}

Wichtig ist das korrekte Zitieren von Quellen, wie es auch von \cite{Kornmeier2011} dargelegt wird. Interessant ist in diesem Zusammenhang auch der Artikel von \cite{Vixie2007}.

\blindtext[4] % Externe Datei einbinden
\chapter{Schreibstil und Typographie}


\section{Hervorhebungen}
\label{Einleitung:Textauszeichnungen}

Achten Sie bitte auf die grundlegenden Regeln der Typographie\index{Typographie}\footnote{Ein Ratgeber in allen Detailfragen ist \cite{Forssman2002}.}, wenn Sie Ihren Text schreiben. Hierzu gehören z.\,B. die Verwendung der richtigen "`Anführungszeichen"' und der Unterschied zwischen Binde- (-), Gedankenstrich (--) und langem Strich (---).

Wenn Sie Text hervorheben wollen, dann setzten Sie ihn \textit{kursiv} (Italic) und nicht \textbf{fett} (Bold). Fettdruck ist Überschriften vorbehalten; im Fließtext stört er den Lesefluss. Das \underline{Unterstreichen} von Fließtext ist im gesamten Dokument tabu und kann maximal bei Pseudo-Code vorkommen.\index{Hervorhebungen}


\section{Anführungszeichen}

Deutsche Anführungszeichen gehen so: "`dieser Text steht in \glq Anführungszeichen\grq; alles klar?"'. Englische Anführungszeichen werden anders benutzt: ``this is an `English' quotation.''


\section{Abkürzungen}
\index{Abkürzungen}
\index{Abbreviation|see{Abkürzungen}}

Eine \ac{ABK} wird bei der ersten Verwendung ausgeschrieben. Danach nicht mehr: \ac{ABK}. Man kann allerdings die Langform explizit anfordern: \acl{ABK} oder die Kurzform \acs{ABK} oder auch noch einmal die Definition: \acf{ABK}.

Beachten Sie, dass bei Abkürzungen, die für zwei Wörter stehen, ein kleines Leerzeichen nach dem Punkt kommt: z.\,B. bzw. \zb, d.\,h. bzw. \dahe.


\section{Querverweise}

Querverweise auf eine Kapitelnummer macht man im Text mit \verb+\ref+ (Kapitel~\ref{Einleitung:Textauszeichnungen}) und auf eine bestimmte Seite mit \verb+\pageref+ (Seite~\pageref{Einleitung:Textauszeichnungen}).


\section{Fußnoten}

Fußnoten werden einfach mit in den Text geschrieben und zwar genau an die Stelle\footnote{An der die Fußnote auftauchen soll.}



\section{Fremdsprachige Begriffe}

Wenn Sie Ihre Arbeit auf Deutsch verfassen, gehen Sie sparsam mit englischen Ausdrücken um. Natürlich brauchen Sie etablierte englische Fachbegriffe, wie z.\,B. \textit{Interrupt}, nicht zu übersetzen. Sie sollten aber immer dann, wenn es einen gleichwertigen deutschen Begriff gibt, diesem den Vorrang geben. Den englischen Begriff (\textit{term}) können Sie dann in Klammern oder in einer Fußnote\footnote{Englisch: \textit{footnote}.} erwähnen. Absolut unakzeptabel sind deutsch gebeugte englische Wörter oder Kompositionen aus deutschen und englischen Wörtern wie z.\,B. downgeloadet, upgedated, Keydruck oder Beautyzentrum. 



\section{Tabellen}

Tabellen werden normalerweise ohne vertikale Striche gesetzt, sondern die Spalten werden durch einen entsprechenden Abstand voneinander getrennt.\footnote{Siehe \cite[S. 89]{Willberg1999}.} Zum Einsatz kommen ausschließlich horizontale Linien (siehe Tabelle~\ref{Kap2:Kopplungsformen}).

\begin{table}[h]
  \caption{Ebenen der Kopplung und Beispiele für enge und lose Kopplung}
  \label{Kap2:Kopplungsformen}
  \renewcommand{\arraystretch}{1.2}
  \centering
  \sffamily
  \begin{footnotesize}
    \begin{tabular}{l l l}
    \toprule
    \textbf{Form der Kopplung} & \textbf{enge Kopplung} & \textbf{lose Kopplung}\\
    \midrule
    Physikalische Verbindung	&	Punkt-zu-Punkt	& 	über Vermittler\\
    Kommunikationsstil	&	synchron		&	asynchron\\
    Datenmodell	&	komplexe gemeinsame Typen	&	nur einfache gemeinsame Typen\\
    Bindung	&	statisch		&	dynamisch\\
    \bottomrule
    \end{tabular}
  \end{footnotesize}
  \rmfamily
\end{table}

Eine Tabelle fließt genauso, wie auch Bilder durch den Text. Siehe Tabelle~\ref{Kap2:Kopplungsformen}.


\section{Aufzählungen}

Aufzählungen sind toll.

\begin{itemize}
  \item Ein wichtiger Punkt
  \item Noch ein wichtiger Punkt
  \item Ein Punkt mit Unterpunkten
    \begin{itemize}
      \item Unterpunkt 1
      \item Unterpunkt 2      
    \end{itemize}
  \item Ein abschließender Punkt ohne Unterpunkte
\end{itemize}


Aufzählungen mit laufenden Nummern sind auch toll.

\begin{enumerate}
  \item Ein wichtiger Punkt
  \item Noch ein wichtiger Punkt
  \item Ein Punkt mit Unterpunkten
    \begin{enumerate}
      \item Unterpunkt 1
      \item Unterpunkt 2      
    \end{enumerate}
  \item Ein abschließender Punkt ohne Unterpunkte
\end{enumerate}


\section{Zitate}

\subsection{Zitate im Text}

Wichtig ist das korrekte Zitieren von Quellen, wie es auch von \cite{Kornmeier2011} dargelegt wird. Interessant ist in diesem Zusammenhang auch der Artikel von \cite{Kramer2009}. Häufig werden die Zitate auch in Klammern gesetzt, wie bei \citep{Kornmeier2011} und mit Seitenzahlen versehen \citep[S. 22--24]{Kornmeier2011}.


\subsection{Zitierstile}

Verwenden Sie eine einheitliche und im gesamten Dokument konsequent durchgehaltene Zitierweise\index{Zitierweise}. Es gibt eine ganze Reihe von unterschiedlichen Standards für das Zitieren und den Aufbau eines Literaturverzeichnisses. Sie können entweder mit Fußnoten oder Kurzbelegen im Text arbeiten. Welches Verfahren Sie einsetzen ist Ihnen überlassen, nur müssen Sie es konsequent durchhalten.

In der Informatik ist das Zitieren mit Kurzbelegen\index{Zitat!Kurzbeleg} im Text (Harvard-Zitierweise) weit verbreitet, wobei für das Literaturverzeichnis häufig die Regeln der \acs{ACM} oder \acs{IEEE} angewandt werden.\footnote{Einen Überblick über viele verschiedene Zitierweisen finden Sie in der \url{http://amath.colorado.edu/documentation/LaTeX/reference/faq/bibstyles.pdf}}

Denken Sie daran, dass das Übernehmen einer fremden Textstelle ohne entsprechenden Hinweis auf die Herkunft in wissenschaftlichen Arbeiten nicht akzeptabel ist und dazu führen kann, dass die Arbeit nicht anerkannt wird. Plagiate\index{Plagiat!Bewertung} werden mit mangelhaft (5,0) bewertet und können weitere rechtliche Schritte nach sich ziehen.


\subsection{Zitieren von Internetquellen}

Internetquellen\index{Zitat!Internetquellen} sind normalerweise \textit{nicht} zitierfähig. Zum einen, weil sie nicht dauerhaft zur Verfügung stehen und damit für den Leser möglicherweise nicht beschaffbar sind und zum anderen, weil häufig der wissenschaftliche Anspruch fehlt.\footnote{Eine lesenswerte Abhandlung zu diesem Thema findet sich (im Internet) bei \cite{Weber2006}}

Wenn ausnahmsweise doch eine Internetquelle zitiert werden muss, z.\,B. weil für eine Arbeit dort Informationen zu einem beschriebenen Unternehmen abgerufen wurden, sind folgende Punkte zu beachten:

\begin{itemize}
\item Die Webseite ist auszudrucken und im Anhang der Arbeit beizufügen,
\item das Datum des Abrufs und die URL sind anzugeben,
\item verwenden Sie Internet-Seiten ausschließlich zu illustrativen Zwecken (z.\,B. um einen Sachverhalt noch etwas genauer zu erläutern), aber nicht zur Faktenvermittlung (z.\,B. um eine Ihrer Thesen zu belegen).
\end{itemize}

Wenn Sie aufgrund der Natur Ihrer Arbeit sehr viele Internetquellen benötigen, dann können Sie diese statt sie auszudrucken auch in elektronischer Form abgeben (CD/DVD). Als Abgabeformat der elektronischen Quellen ist PDF/A\footnote{Bei PDF/A handelt es sich um ein besonders stabile Variante des \ac{PDF}, die von der  \ac{ISO} standardisiert wurde.} vorteilhaft, weil es von allen Formaten die größte Stabilität besitzt.
Auf der CD/DVD geben Sie bitte auch eine HTML-Version des Literaturverzeichnisses ab, in der die Online-Quellen sowie die gespeicherten PDF-Dateien verlinkt sind.

Wikipedia\index{Zitat!Wikipedia} stellt einen immensen Wissensfundus dar und enthält zu vielen Themen hervorragende Artikel. Sie müssen sich aber darüber im Klaren sein, dass die Artikel in Wikipedia einem ständigen Wandel unterworfen sind und nicht als Quelle für wissenschaftliche Fakten genutzt werden sollten. Es gelten die allgemeinen Regeln für das Zitieren von Internetquellen. Sollten Sie doch Wikipedia nutzen müssen, verwenden Sie bitte ausschließlich den Perma-Link\footnote{Sie erhalten den Permalink über die Historie der Seite und einen Klick auf das Datum.}\index{Permalink} zu der Version der Seite, die Sie aufgerufen haben.


 % Externe Datei einbinden
\chapter{Einbinden von Grafiken, Sourcecode und Anforderungen}
\label{Kap3}

\section{Bilder}

Natürlich können auch Grafiken und Bilder eingebunden werden, siehe z.\,B. Abbildung~\ref{Kap2:NasaRover}.

\begin{figure}[ht]
  \centering
  \includegraphics[width=6cm]{kapitel3/nasa_rover}
  \caption{Ein Nasa Rover}
  \label{Kap2:NasaRover}
\end{figure}

Man kann sich auch selbst ein Makro für das Einfügen von Bildern schreiben:

\bild{kapitel3/modell_point_to_point}{6cm}{Point to Point}

\begin{sidewaysfigure}
 \includegraphics[width=22cm]{kapitel3/ws-wsdl20-fehler}
  \caption{Sehr große Grafiken kann man drehen, damit sie auf die Seite passen}
  \label{Kap2:wsdl-fehler}
\end{sidewaysfigure}

\clearpage % Alle Bilder, die bisher kamen ausgeben


\section{Formelsatz}

Eine Formel gefällig? Mitten im Text $a_2 = \sqrt{x^3}$ oder als eigener Absatz (siehe Formel~\ref{Formel}):

\begin{equation}
\begin{bmatrix}
   1 &  4 &  2 \\
   4 &  0 & -3
\end{bmatrix}
        \cdot
\begin{bmatrix}
   1 &  1 &  0 \\
  -2 &  3 &  5 \\
   0 &  1 &  4
\end{bmatrix}
       {=}
\begin{bmatrix}
  -7 &  15 &  28 \\
   4 &   1 & -12
\end{bmatrix}
\label{Formel}
\end{equation}


\section{Sourcecode}

Man kann mit Latex auch ganz toll Sourcecode in den Text aufnehmen.

\subsection{Aus einer Datei}

\lstinputlisting[firstline=2,language=Java,caption={Crypter-Interface},label=lst:CrypterInterface]{\srcloc/Crypter.java}


\subsection{Inline}

\begin{lstlisting}[language=Java,caption=Methode checkKey()]
    /**
     * Testet den Schlüssel auf Korrektheit: Er muss mindestens die Länge 1
     * haben und darf nur Zeichen von A-Z enthalten.
     *
     * @param key zu testender Schlüssel
     * @throws CrypterException wenn der Schlüssel nicht OK ist.
     */
    protected void checkKey(Key key) throws CrypterException {

        // Passt die Länge?
        if (key.getKey().length == 0) {
            throw new CrypterException("Der Schlüssel muss mindestens " +
                    "ein Zeichen lang sein");
        }

        checkCharacters(key.getKey(), ALPHABET);
    }
\end{lstlisting}


\section{Anforderungen}

Anforderungen im Format des Volere"=Templates (Snowcards) \autocite{Volere} können per Makro eingefügt werden. Das Label wird automatisch mit der Nummer erstellt, d.\,h. Sie können auf die Tabelle mit dieser referenzieren (siehe \autoref{F52}).

\snowcard % Snowcard einbinden (Anpassungen in titelblatt.tex)
   {F52} % Nummer des Requirements
   {F} % Art
   {Hoch} % Priorität
   {User Authentifizierung} % Titel
   {Interview mit Abteilungsleiter} % Herkunft (Optional)
   {F12} % Konflikte (Optional)
   {Der Benutzer ist in der Lage sich über seinen
    Benutzernamen und sein Passwort am System anzumelden} % Beschreibung
   {Ein Benutzer kann sich mit seinem firmenweiten Benutzernamen und
   Passwort über die Anmeldemaske anmelden und hat Zugriff auf die
   Funktionen des Systems} % Fit-Kriterium (Optional)
   {Benutzerhandbuch des Altsystems} % Material (Optional)

Ebenso können Sie nicht"=funktionale Anforderungen mit Hilfe von Quality Attribute Scenarios (vgl. \autoref{NF11}) darstellen. Zu Details siehe \autocite{Barbacci2003}.

\qas % Quality-Attribute Scenario einbinden (Anpassungen in titelblatt.tex)
   {NF11} % Nummer des Requirements
   {Hoch} % Priotität
   {Performance des Jahresabschlusses} % Titel
   {Endbenutzer} % Quelle
   {Startet einen Jahresabschluss} % Stimulus
   {Buchhaltungssystem} % Artefakt
   {Das System befindet sich im normalen Betriebszustand} % Umgebung
   {Jahresabschluss ist durchgeführt und kann als PDF abgerufen werden} % Antwort
   {10 Minuten} % Antwort-Maß

Die Abgrenzung von funktionalen und nicht-funktionalen Anforderungen ist nicht immer einfach und bereitet manchen Studierenden Probleme. Als Hilfestellung kann die von der ISO25010 \autocite{ISO25010} zur Verfügung gestellte Liste dienen, siehe \autoref{kapitel3/iso25010}.

\bild{kapitel3/iso25010}{14cm}{Qualitätsmodell für Software-Produkte nach ISO25010}

\citeauthor{Bass2003} listen in \autocite{Bass2003} eine ähnliche Liste von Kategorien für nicht-funktionalen Anforderungen auf, die ebenfalls als Richtschnur dienen kann. Diese sind:

\begin{itemize}
  \item \textit{Verfügbarkeit} \textit{(availability)} -- umfasst Zuverlässigkeit (reliability), Robustheit (robustness), Fehlertoleranz (fault tolerance) und Skalierbarkeit (scalability)
  \item \textit{Anpassbarkeit} \textit{(modifiability)}, umfasst Wartbarkeit (maintainability), Verständlichkeit (understandability) und Portabilität (portability).
  \item \textit{Performanz} \textit{(performance)}
  \item \textit{Sicherheit} \textit{(security)}
  \item \textit{Testbarkeit} \textit{(testability)}
  \item \textit{Bedienbarkeit} \textit{(usability)}
\end{itemize}
 % Externe Datei einbinden
\chapter{Checkliste}
\label{Kap4}

Die folgende Checkliste kann dazu dienen, die Arbeit auf die wichtigsten Bewertungskriterien zu prüfen. Jeder Dozent hat andere Kriterien, die unten aufgeführten dürften aber für die meisten Dozenten gültig sein.

\section{Form und Sprache}

\begin{checklist}
  \footnotesize
  \item \textbf{Aufbau}: Die Arbeit ist nach wissenschaftlichen Prinzipien aufgebaut (wesentliche Teile vorhanden, Nummerierung/Verweise korrekt, Verzeichnisse vorhanden).
    \begin{checklist}
        \item \textit{Wesentliche Teile}: Die folgenden Elemente der Arbeit sind vorhanden: Titelblatt, Abstract/Zusammenfassung, Einleitung, Hauptteil, Fazit/Ausblick.
        \item \textit{Nummerierung/Verweise}: Das Nummerierungsschema wird konsistent über die gesamte Arbeit durchgehalten, die Verweise auf die verschiedenen Elemente (Abbildungen, Tabellen etc.) sind korrekt.
        \item \textit{Verzeichnisse}: Die Arbeit enthält alle relevanten Verzeichnisse: Inhaltsverzeichnis, Literaturverzeichnis, Abbildungsverzeichnis, Tabellenverzeichnis, eventuell Glossar.
    \end{checklist}
  \item \textbf{Sprache}: Die verwendete Sprache entspricht wissenschaftlichen Ansprüchen.
    \begin{checklist}
        \item \textit{Begriffe und Definitionen}: Begriffe werden einheitlich und konsistent verwendet. Neue Begriffe werden definiert und mit Literatur hinterlegt.
        \item \textit{Abkürzungen}: Alle Abkürzungen werden eingeführt und erläutert. Abkürzungen werden bei der ersten Verwendung ausgeschrieben und in einem Abkürzungsverzeichnis geführt. Es werden keine unüblichen oder selbst erfunden Abkürzungen verwendet. Ein Glossar kann verwendet werden, um Begriffe noch einmal kompakt darzustellen.
        \item \textit{Rechtschreibung}: Die Arbeit ist frei von Rechtschreibungs-, Zeichensetzungs- und Grammatikfehlern.
    \end{checklist}
  \item \textbf{Formatierung, Typografie}: Die Formatierung der Arbeit ist korrekt und aus typographischer Sicht einwandfrei. \textit{Wenn Sie dieses Template korrekt verwenden, sollte dieser Punkt automatisch durch die Verwendung von \LaTeX \ erledigt sein.}
    \begin{checklist}
        \item \textit{Korrekte Typografie}: Schriftarten werden korrekt verwendet (nicht mehr als 2 Fonts), der Zeilenabstand ist passend, die Ränder sind ausreichend, der Satz ist korrekt.
        \item \textit{Satz von Abbildungen, Tabellen etc.}: Abbildungen sind in der richtigen Auflösung dargestellt, die Tabellen sind korrekt gesetzt, mathematische Formeln und Symbole sind sauber dargestellt.
    \end{checklist}
  \item \textbf{Abbildungen}: Abbildungen werden in ausreichendem Umfang zur Förderung des Verständnisses eingesetzt. Sie werden korrekt im Text referenziert und sind, wo immer möglich, in einer Standardnotation erstellt.
    \begin{checklist}
        \item \textit{Ausreichende Verwendung}: Komplizierte Sachverhalte werden durch Abbildungen verdeutlicht. Es werden genug Abbildungen eingesetzt, um die wichtigsten Sachverhalte zu erklären.
        \item \textit{Verständnisförderung}: Abbildungen dienen nicht als Schmuck, sondern um komplizierte Sachverhalte zu verdeutlichen.
        \item \textit{Einbindung in den Text}: Der Text muss auch ohne Abbildungen verständlich sein, die Abbildungen helfen Sachverhalte aus dem Text besser darzustellen. Der Text referenziert die Abbildung korrekt.
        \item \textit{Standardnotation, Legende}: Die Abbildungen verwenden Standard"=Notationen wie UML, FMC etc. Wo keine Standardnotation eingesetzt wird, ist eine Legende vorhanden, um die Bildelemente zu erläutern.
    \end{checklist}
  \item \textbf{Zitate}: Quellen werden konsistent nach einer gängigen Zitierweise zitiert und sind vollständig im Literaturverzeichnis angegeben.
    \begin{checklist}
        \item \textit{Zitierweise}: Die Zitierweise in der gesamten Arbeit folgt einem einheitlichen Schema, z.B. IEEE, DIN, Chicago.
        \item \textit{Vollständigkeit}: Alle Zitate sind als solche kenntlich gemacht und die Quelle wird vollständig angegeben, und Plagiate werden vermieden.
    \end{checklist}
  \item \textbf{Schreibstil}: Lebendiger, wissenschaftlicher und verständlicher Schreibstil.
    \begin{checklist}
        \item \textit{Wissenschaftlichkeit}: Der Text ist im Präsenz geschrieben, es wird die dritte Person verwendet, Fachausdrücke werden korrekt verwendet, Fremdwörter und Amerikanismen werden richtig eingesetzt.
        \item \textit{Verständlichkeit}: Abschweifungen und Wiederholungen werden vermieden, statt dessen werden präzise und übersichtliche Sätze verwendet.
        \item \textit{Lebendigkeit}: Der Text der Arbeit zeichnet sich durch eine gute Wortwahl, Sprachbilder, einen angemessenen Satzbau und eine hohe Variabilität aus.
    \end{checklist}
\end{checklist}

\section{Inhalt}

\begin{checklist}
  \footnotesize
  \item \textbf{Gliederung}: Die Gliederung ist vollständig, konsistent und sachlogisch mit angemessener Struktur und Tiefe.
    \begin{checklist}
        \item \textit{Konsistenz und Vollständigkeit}: Auf einer Ebene stehen keine Punkte alleine, die Gliederungspunkte orientieren sich an der Argumentationskette.
        \item \textit{Angemessene Tiefe}: Die Größe der einzelnen Unterpunkte ist vom Umfang her ähnlich. Es gibt keine Gliederungspunkte, die nur aus ein bis zwei Sätzen bestehen.
    \end{checklist}
  \item \textbf{Grundlagen}: Es werden alle relevanten Grundlagen gelegt. Der State"=of"=the"=art und der State"=of"=practice werden dargelegt.
    \begin{checklist}
        \item \textit{Umfang}: 1/3 des Hauptteils ist ein gutes Maß für eine ausreichende Darstellung der Grundlagen.
        \item \textit{Begriffe und Methoden}: Begriffe und Methoden sind definiert, und Literatur zu den Definitionen ist angegeben.
        \item \textit{State-of-the-art}: Der Stand des verfügbaren Wissens wird dargestellt, analysiert und kritisch beurteilt (state-of-the-art). Bei theoretischen Arbeiten kann ein eigenes Kapitel \enquote{verwandte Arbeiten} nötig sein, um den state"=of"=the"=art darzustellen.
        \item \textit{State-of-practice}: Bei praktischen Arbeiten, die in der Industrie geschrieben werden, kann es nötig sein, auch das Vorgehen im Unternehmen zu erläutern.
    \end{checklist}
  \item \textbf{Methodik/Lösung}: Die gewählte Methodik bzw. Lösung ist für das Problem adäquat.
    \begin{checklist}
        \item \textit{Anforderungen an die Lösung}: Die von der Lösung zu erfüllenden Anforderungen werden dargestellt. Wo nötig wird dies auf Grundlage eines sauberen Requirements"=Engineerings durchgeführt.
        \item \textit{Erläuterung des Lösungsansatzes}: Der gewählte Lösungsansatz wird ausführlich erläutert und verständlich dargestellt.
        \item \textit{Eignung zur Lösung der Aufgabe}: Die gewählte Lösung ist geeignet, um das beschriebene Problem zu lösen.
        \item \textit{Hypothesen}: Es sind ggf. Hypothesen gebildet worden; diese sind erläutert, und es sind Kriterien identifiziert worden, mit deren Hilfe man die Hypothesen falsifizieren kann.
        \item \textit{Alternativen}: Es werden Alternativen zur vorgeschlagenen Lösung diskutiert. Die eigene Lösung wird nicht als einzige mögliche dargestellt, sondern es werden auch andere mögliche Lösungen vorgestellt und bewertet.
        \item \textit{Begründung}: Alternativen und Kriterien für die Auswahl dieser Lösung werden dargestellt.
        \item \textit{Vorteile der Lösung}: Es wird dargestellt, wieso die entwickelte Lösung vorteilhafter ist als die bisherigen Ansätze. Diese Darstellung erfolgt auf Basis des Lösungsansatzes. Eine konkrete Validierung der Implementierung erfolgt ggf. in späteren Kapiteln.
    \end{checklist}
  \item \textbf{Logik der Argumentationskette}: Die Argumentation ist logisch und nachvollziehbar. Sie ist frei von logischen Fehlschlüssen.
  \item \textbf{Implementierung}: Wenn eine Implementierung der Lösung erfolgt, so wird die Implementierung beschrieben. Die Darstellung der Implementierung kann knapp ausfallen. Wichtig ist der Lösungsansatz, nicht die konkrete Umsetzung.
  \item \textbf{Validierung}: Die vorgeschlagene Lösung wird ggf. empirisch verprobt.
    \begin{checklist}
        \item \textit{Vorgehensweise}: Die Vorgehensweise zur Validierung der Lösung / Hypothesen ist beschrieben und geeignet, relevante Aspekte der Lösung zu überprüfen.
        \item \textit{Empirische Analyse}: Die Erfassungsmethode wird dargestellt und die Daten werden nach den Grundsätzen ordnungsgemäßer Laborpraxis gesammelt und statistisch korrekt ausgewertet.
        \item \textit{Verprobung}: Die Lösung wird an einem praktischen Beispiel verprobt, und es werden wissenschaftlich korrekte Schlüsse aus der Anwendung gezogen.
        \item \textit{Zielerreichung}: Funktioniert die gewählte Lösung nach der Implementierung? Wie weit wurde das Ziel erreicht? Falls nicht, gibt es nachvollziehbare Gründe dafür und wurden diese dargestellt?
    \end{checklist}
  \item \textbf{Diskussion}: Die Lösung und ihre Validierung wird kritisch und im Kontext möglicher Alternativen diskutiert und bewertet.
    \begin{checklist}
        \item \textit{Kritische Reflexion}: Grenzen und Schwächen der eigenen Ergebnisse werden beleuchtet.
        \item \textit{Ableitung von Konsequenzen}: Die Konsequenzen aus den Ergebnissen für die Wissenschaft und Praxis sind beschrieben.
    \end{checklist}
  \item \textbf{Quellenarbeit}: Es werden hochwertige Quellen in ausreichendem Umfang genutzt und kritisch hinterfragt. Eventuell vorhandene Quellen aus dem Unternehmen werden ebenfalls berücksichtigt.
    \begin{checklist}
        \item \textit{Umfang}: Der Umfang an Quellen richtet sich stark nach Thema und Art der Arbeit. Bei einer Bachelorarbeit sind mindestens 20--30 Quellen üblich, bei einer Masterarbeit deutlich mehr.
        \item \textit{Wissenschaftliche Qualität}: Nicht zitierfähig sind Internet"=Quellen, Wikipedia"=Einträge sowie andere Bachelor- oder Masterarbeiten (sofern nicht veröffentlicht). Das ausschließliche Zitieren von Lehrbüchern ist problematisch. Aktuelle wissenschaftliche Artikel und Werke sollten in den Quellen auftauchen.
        \item \textit{Quellen \enquote{aus der Praxis}}: Wenn es im Unternehmen spezielle Quellen und Informationen gibt, so werden diese berücksichtigt, z. B. firmen- oder branchenspezifischer Informationen.
        \item \textit{Kritische Würdigung}: Quellen und Zitate werden kritisch hinterfragt und nicht einfach unreflektiert übernommen. Es gibt eine kritische Distanz bei der Quellenauswahl und Quellenauswertung.
    \end{checklist}
  \item \textbf{Fazit}: Es wird eine Zusammenfassung der Arbeit sowie Ausblick auf weitere mögliche Arbeiten im Themenfeld gegeben, etwa die Lösung ausstehender Probleme oder die Erfüllung zusätzlicher Anforderungen.
  \item \textbf{Umfang der Arbeit}: Richtgrößen: Bachelorarbeiten: 50--80 Seiten, Masterarbeiten: 60--100 Seiten, jeweils ohne Verzeichnisse und Anhang.
\end{checklist}

\section{Vor der Abgabe}

\begin{checklist}
  \footnotesize
  \item \textit{Korrektur}: Haben Sie einen Dritten die Arbeit lesen lassen und alle gefundenen Rechtschreib- und Zeichensetzungsfehler behoben?
  \item \textit{Literaturverzeichnis}: Sind im Literaturverzeichnis irrelevante Informationen entfernt? Beispielsweise bei Büchern unnötige Informationen über die Herkunft bei Google-Books oder bei Papern doppelte Angaben der DOI?
  \item \textbf{Abgabe auf Papier}
  \begin{checklist}
    \item \textit{Template passend eingestellt}: Haben Sie in der Datei \texttt{thesis.tex} eingestellt, dass Sie auf Papier abgeben wollen?
    \item \textit{Doppel- oder einseitiger Druck}: Entspricht die Einstellung des Templates dem Druck, d.\,h. ist das Template für doppelseitigen Druck eingestellt, wenn doppelseitig gedruckt werden soll und umgekehrt?
    \item \textit{Umschläge}: Sind die Umschläge vorhanden, um die Arbeit später zu binden? Die Umschläge können in der Hausdruckerei der Hochschule erworben werden.
    \item \textit{Copyshop}: Wissen Sie, wo Sie die Arbeit drucken werden? Die Hausdruckerei kann Ihre Arbeit nicht drucken.
    \item \textit{Exemplare}: Haben Sie geklärt, ob der Zweitkorrektor auch ein gedrucktes Exemplar möchte?
  \end{checklist}
  \item \textbf{Digitale Abgabe}
  \begin{checklist}
    \item \textit{Zustimmung des Betreuers/der Betreuerin}: Haben Sie mit Ihrer Betreuerin bzw. Ihrem Betreuer abgeklärt, dass Sie digital abgeben dürfen?
    \item \textit{Template passend eingestellt}: Haben Sie in der Datei \texttt{thesis.tex} eingestellt, dass Sie digital abgeben wollen?
    \item \textit{Unterschrift}: Haben Sie Ihre Unterschrift eingescannt und unter dem Namen \texttt{unterschrift.png} im Hauptverzeichnis abgelegt?
  \end{checklist}
\end{checklist}
 % Externe Datei einbinden



\label{lastpage}
%Beginn des Anhangs. Befehl \appendix nicht entfernen auch wenn kein Anhang vorhanden ist!
\appendix

%Wenn Sie keinen Anhang haben, entfernen Sie ausschließlich die nachfolgenden beiden Dateien.
\chapter{Erster Anhang}

Hier ein Beispiel für einen Anhang. Der Anhang kann genauso in Kapitel und Unterkapitel unterteilt werden, wie die anderen Teile der Arbeit auch.

\chapter{Zweiter Anhang}

Hier noch ein Beispiel für einen Anhang.



\end{document}

