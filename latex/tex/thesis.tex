% *******************************************************************
% STOP - Bitte zuerst lesen, bevor Sie weitermachen
%
% Einige Dinge müssen Sie an Ihre Bedürfnisse (und die Vorgaben Ihres
% Betreuers anpassen).
%
% 1. Sprache
% Das Template unterstützt Deutsch und Englisch, Standard ist Deutsch.
% Wenn Sie Englisch verwenden wollen, ändern Sie bitte
%   a) in docinfo.tex \hsmasprache auf en
%   b) in preambel.tex \usepackage[main=english, ngerman]{babel},
%      \usepackage[pagebackref=false,english]{hyperref},
%      \usepackage[autostyle=true,english=quotes]{csquotes} und bei
%      der Literatur die Sortierreihenfolge mit sortlocale=en_US.
%
% 2. Zitierstil
% Abhängig von dem gewünschten Zitierstil passen Sie bitte in
% preambel.tex die Einstellungen bei \usepackage[backend=biber...
% an. Wie ist dort genau erklärt.
%
% 3. Doppelseitiger oder einseitiger Druck
% Das Template ist für doppelseitigen Druck eingestellt. Wollen Sie
% einseitig drucken, müssen Sie in preambel.tex die Einstellungen
% für \documentclass ändern. Und zwar von twoside=on auf twoside=off.
%
% 4. Abkürzungen auf richtige Breite einstellen
% In der Datei kapitel/abkuerzungen.tex müssen Sie die _längste_
% Abkürzung in die eckigen Klammern von \begin{acronym} schreiben,
% sonst werden die Abkürzungen nicht richtig ausgerichtet.
% Also z.B. \begin{acronym}[DSGVO].
%
% 5. Unnötige Teile entfernen
% Entfernen Sie die Teile, die Sie nicht brauchen, z.B. Anhänge,
% Quelltextverzeichnis etc. Siehe unten
%
% 6. Silbentrennung
% LaTeX führt eine automatische Silbentrennnung durch. Allerdings
% werden Wörter, die bereits einen Bindestrich enthalten nicht
% getrennt, z.B. Datenschutz-Grundverordnung. Wenn Sie Ihren Text auf
% Deutsch schreiben, können Sie dann alternativ "= für den Bindestrich
% im Wort verwenden, z.B. Datenschutz"=Grundverordnung, damit LaTeX
% weiterhin richtig trennt.
% Ist die Silbentrennung aus einem anderen Grund nicht erfolgt, sodass
% das Wort über den rechten Rand hinaussteht oder wenn Sie eine weitere
% Trennstelle wollen, können Sie LaTeX helfen, indem Sie weitere
% Trennstellen angeben. Dies geschieht durch "- als Zeichen, z.B.
% Staats"-vertrag.
% *******************************************************************


% Preambel mit Einstellungen importieren
% Document type and used packages
\documentclass[open=right, % Kapitel darf nur auf rechten Seite beginnen
paper=A4,               % DIN-A4-Papier
a4paper,                % DIN-A4-Papier
12pt,                   % Schriftgöße
headings=small,         % Kleine Überschriften
headsepline=true,       % Trennlinie am Kopf der Seite
footsepline=false,      % Trennlinie am Fuß der Seite
bibliography=totoc,     % Literaturverzeichnis in das Inhaltsverzeichnis aufnehmen
twoside=on,             % Doppelseitiger Druck - auf off stellen für einseitig
DIV=7,
cleardoublepage=plain]{scrbook} 

% Pakete einbinden, die benötigt werden
\usepackage[utf8]{inputenc}       % Dateien in UTF-8 benutzen
\usepackage[T1]{fontenc}          % Zeichenkodierung
\usepackage{graphicx}             % Bilder einbinden
\usepackage[ngerman,english]{babel}       % Deutsch und Englisch unterstützen
\usepackage{xcolor}               % Color support
\usepackage{amsmath}              % Matheamtische Formeln
\usepackage{amsfonts}             % Mathematische Zeichensätze
\usepackage{amssymb}              % Mathematische Symbole
\usepackage{float}                % Fließende Objekte (Tabellen, Grafiken etc.)
\usepackage{booktabs}             % Korrekter Tabellensatz
\usepackage[printonlyused]{acronym}   % Abkürzungsverzeichnis [nur verwendete Abkürzugen]
\usepackage{makeidx}              % Sachregister
\usepackage{fancyhdr}             % Schönere Überschriften
\usepackage{listings}             % Source Code listings
\usepackage{listingsutf8}         % Listings in UTF8
\usepackage[hang,font={sf,footnotesize},labelfont={footnotesize,bf}]{caption} % Beschriftungen
\usepackage[scaled]{helvet}       % Schrift Helvetia laden
\usepackage[sf,bf,small]{titlesec} % Einstellungen für Überschriften
\usepackage[absolute]{textpos}	% Absolute Textpositionen (für Deckblatt)
\usepackage{calc}                 % Berechnung von Positionen
\usepackage{blindtext}            % Blindtexte
\usepackage[bottom=40mm,left=35mm,right=35mm,top=30mm]{geometry} % Ränder ändern
\usepackage[square]{natbib}       % Literaturverzeichnis nach DIN mit eckigen Klammern bei \citep
\usepackage{setspace}             % Abstände korrigieren
\usepackage{ifthen}               % Logische Bedingungen mit ifthenelse
\usepackage{scrhack}              % Get rid of tocbasic warnings
\usepackage[pagebackref=false]{hyperref}  % Hyperlinks
\usepackage[all]{hypcap}          % Korrekte Verlinkung von Floats

% Farben definieren
\definecolor{linkblue}{RGB}{0, 0, 100}
\definecolor{linkblack}{RGB}{0, 0, 0}
\definecolor{comment}{RGB}{63, 127, 95}
\definecolor{darkgreen}{RGB}{14, 144, 102}
\definecolor{darkblue}{RGB}{0,0,168}
\definecolor{darkred}{RGB}{128,0,0}
\definecolor{javadoccomment}{RGB}{0,0,240}

% Einstellungen für das Hyperlink-Paket
\hypersetup{
    colorlinks=true,      % Farbige links verwenden       
%    allcolors=linkblue,
    linktoc=all,          % Links im Inhaltsverzeichnis
    linkcolor=linkblack,  % Querverweise
    citecolor=linkblack,  % Literaturangaben
	filecolor=linkblack,  % Dateilinks
	urlcolor	=linkblack    % URLs
}

% Einstellungen für Quelltexte
\lstset{     
      xleftmargin=0.2cm,     
      basicstyle=\footnotesize\ttfamily,
      keywordstyle=\color{darkgreen},
      identifierstyle=\color{darkblue},
      commentstyle=\color{comment}, 
      stringstyle=\color{darkred}, 
      tabsize=2,
      lineskip={2pt},
      columns=flexible,
      inputencoding=utf8,
      captionpos=b,
      breakautoindent=true,
	  breakindent=2em,
	  breaklines=true,
	  prebreak=,
	  postbreak=,
      numbers=none,
      numberstyle=\tiny,
      showspaces=false,      % Keine Leerzeichensymbole
      showtabs=false,        % Keine Tabsymbole
      showstringspaces=false,% Leerzeichen in Strings
      morecomment=[s][\color{javadoccomment}]{/**}{*/},
      literate={Ö}{{\"O}}1 {Ä}{{\"A}}1 {Ü}{{\"U}}1 {ß}{{\ss}}2 {ü}{{\"u}}1 {ä}{{\"a}}1 {ö}{{\"o}}1
}

\urlstyle{same}

% Einstellungen für Überschriften
\titlespacing{\paragraph}{0pt}{1ex}{2.0ex}
\titlespacing{\subsubsection}{0pt}{3ex}{0.0ex}
\titlespacing{\subsection}{0pt}{4ex}{0.2ex}
\titlespacing{\section}{0pt}{7ex}{1ex}
\titleformat*{\subsubsection}{\sffamily\itshape\bfseries\small}
\titleformat*{\paragraph}{\sffamily\bfseries\small}

% Einstellungen für Schriftarten
\setkomafont{pagehead}{\normalfont\sffamily}
\setkomafont{pagenumber}{\normalfont\sffamily}
\addtokomafont{footnote}{\footnotesize}

% Wichtige Abstände
\setlength{\parskip}{0.2cm}  % 2mm Abstand zwischen zwei Absätzen
\setlength{\parindent}{0mm}  % Absätze nicht einziehen
\clubpenalty = 10000         % Keine "Schusterjungen"
\widowpenalty = 10000        % Keine "Hurenkinder"
\displaywidowpenalty = 10000 % Keine "Hurenkinder"
\renewcommand{\footnotesize}{\fontsize{9}{10}\selectfont} % Größe der Fußnoten
\setlength{\footnotesep}{8pt} % Abstand zwischen den Fußnoten

% Index erzeugen
\makeindex

% Einfacher Font-Wechsel über dieses Makro
\newcommand{\changefont}[3]{
\fontfamily{#1} \fontseries{#2} \fontshape{#3} \selectfont}

% Eigenes Makro für Bilder
\newcommand{\bild}[3]{
\begin{figure}[h]
  \centering
  \includegraphics[width=#2]{#1}
  \caption{#3}
  \label{#1}
\end{figure}}

% Wo liegt Sourcecode?
\newcommand{\srcloc}{src/}

% Wo sind die Bilder?
\graphicspath{{bilder/}}

% Makros für typographisch korrekte Abkürzungen
\newcommand{\zb}[0]{z.\,B.\ }
\newcommand{\dahe}[0]{d.\,h.\ }
\newcommand{\ua}[0]{u.\,a.\ }


% Dokumenteninfos importieren
% -------------------------------------------------------
% Informationen und Einstellungen für Ihre Abschlussarbeit
%

% Sprache für das Dokument festlegen
\newcommand{\hsmasprache}{de} %de für Deutsch oder en für Englisch


% Abgabeform festlegen
% Bei einer digitalen Abgabe, wird das Dokument einseitig erzeugt und der Titel wird
% zentriert.
\newcommand{\hsmaabgabe}{digital} % Abgabe erfolgt für Fakultät I digital. Optionen hier sind für anderen Fakultäten: "papier" oder "digital".


% Flags für Veröffentlichung, Sperrvermerk
\newcommand{\hsmapublizieren}{opensource}   
% Wird einer Veröffentlichung zugestimmt?
% Optionen: 
% opensource = Druck der CC Lizenz mit By SA (Standard)
% hs = Veröffentlichung an der Technischen Hochschule und auf Hochschulservern
% stud = kein opensource und keine veröffentlichung auf den Hochschulservern
% vertraulich = Arbeit darf nicht veröffentlicht werden und erhält einen Sperrvermerk (Nur nach Absprache mit Betreuer setzen!)


\newcommand{\genderhinweis}{gender}     % Soll der Gender-Hinweis angezeigt werden? ja=gender, nein = nogender; Genderhinweis wird nur in deutscher Sprache angezeigt!


\newcommand{\hsmaquellcode}{sourcecode} % Verwenden Sie Quellcode in Ihrer Arbeit? ja=sourcecode, nein= nosourcecode

\newcommand{\hsmasymbole}{symbole} % Verwenden Sie viele Symbole in Ihrer Arbeit, welche in einem Symbolverzeichnis aufgeführt werden sollen? ja=symbole, nein= nosymbole


\newcommand{\hsmaglossar}{glossar} % Verwenden Sie Begriffserklärungen nicht Abkürzungen in Ihrer Arbeit? ja=glossar, nein= noglossar

\newcommand{\hsmatc}{tc} % Verwenden der Änderungsmarkierung. Änderungsmarkierung aktiv und eine Liste der Änderungen wird angezeigt = tc, Keine Änderungsmarkierung und keine Ausgabe der Änderungen = notc




% Titel der Arbeit auf Deutsch
\newcommand{\hsmatitelde}{Einsatz eines Flux-Kompensators für Zeitreisen mit einer maximalen Höchstgeschwindigkeit von WARP~7}

% Titel der Arbeit auf Englisch
\newcommand{\hsmatitelen}{Application of a flux compensator for timetravel with a maximum velocity of warp~7}

% Weitere Informationen zur Arbeit
\newcommand{\hsmaort}{Mannheim}          % Ort
\newcommand{\hsmaautorvname}{Max}        % Vorname(n)
\newcommand{\hsmaautornname}{Mustermann} % Nachname(n)
\newcommand{\hsmaabgabedatum}{2025-03-04}% Datum der Abgabe in dem Format JJJJ-MM-TT

\newcommand{\hsmafirma}{Paukenschlag GmbH, Mannheim} % Firma bei der die Arbeit durchgeführt wurde
\newcommand{\hsmabetreuer}{Prof. Peter Mustermann, Technische Hochschule Mannheim} % Betreuer an der Hochschule
\newcommand{\hsmazweitkorrektor}{Erika Mustermann, Paukenschlag GmbH}   % Betreuer im Unternehmen oder Zweitkorrektor

\newcommand{\hsmafakultaet}{I}    % I für Informatik oder E, S, B, D, M, N, W, V
\newcommand{\hsmastudiengang}{IB} % IB IMB UIB CSB IM MTB (weitere siehe titleblatt.tex)


% Literatur-Datenbank
\addbibresource{literatur.bib}   % BibLaTeX-Datei mit Literaturquellen einbinden

\begin{document}
\frontmatter

% Römische Ziffern für die "Front-Matter"
\setcounter{page}{0}
\changefont{ptm}{m}{n}  % Times New Roman für den Fließtext
\renewcommand{\rmdefault}{ptm}

% Titelblatt
% -------------------------------------------------------
% In dieser Datei sollten eigentlich keine Veränderungen mehr
% notwendig sein.
% -------------------------------------------------------

\thispagestyle{empty}

\ifthenelse{\equal{\hsmafakultaet}{I}}%
  {\newcommand{\hsmafakultaetlangde}{Fakultät für Informatik}%
   \newcommand{\hsmafakultaetlangen}{Department of Computer Science}}{}

\ifthenelse{\equal{\hsmastudiengang}{IB}}%
  {\newcommand{\hsmastudienganglangde}{Informatik}%
  \newcommand{\hsmastudienganglangen}{Computer Science}%
  \newcommand{\hsmatypde}{Bachelor-Thesis}%
  \newcommand{\hsmatypen}{Bachelor Thesis}%
  \newcommand{\hsmagrad}{\hsmabsc}}{}

\ifthenelse{\equal{\hsmastudiengang}{IMB}}%
  {\newcommand{\hsmastudienganglangde}{Medizinische Informatik}%
  \newcommand{\hsmastudienganglangen}{Medical Informatics}%
  \newcommand{\hsmatypde}{Bachelor-Thesis}%
  \newcommand{\hsmatypen}{Bachelor Thesis}%
  \newcommand{\hsmagrad}{\hsmabsc}}{}
  
\ifthenelse{\equal{\hsmastudiengang}{UIB}}%
  {\newcommand{\hsmastudienganglangde}{Unternehmens- und Wirtschaftsinformatik}%
  \newcommand{\hsmastudienganglangen}{Enterprise Computing}%  
  \newcommand{\hsmatypde}{Bachelor-Thesis}%
  \newcommand{\hsmatypen}{Bachelor Thesis}%
  \newcommand{\hsmagrad}{\hsmabsc}}{}

\ifthenelse{\equal{\hsmastudiengang}{IM}}%
  {\newcommand{\hsmastudienganglangde}{Informatik}%
   \newcommand{\hsmastudienganglangen}{Computer Science}%
   \newcommand{\hsmatypde}{Master-Thesis}%
   \newcommand{\hsmatypen}{Master Thesis}%
   \newcommand{\hsmagrad}{\hsmamaster}}{}

\ifthenelse{\equal{\hsmastudiengang}{MTB}}%
  {\newcommand{\hsmastudienganglangde}{Mechatronik}%
   \newcommand{\hsmastudienganglangen}{Mechatronics}%
   \newcommand{\hsmatypde}{Bachelor-Thesis}%
   \newcommand{\hsmatypen}{Bachelor Thesis}%
   \newcommand{\hsmagrad}{\hsmabsc}}{}

\newcommand{\hsmabsc}{Bachelor of Science (B.Sc.)}
\newcommand{\hsmamaster}{Master of Science (M.Sc.)}

\newcommand{\hsmakoerperschaftde}{Hochschule Mannheim}
\newcommand{\hsmakoerperschaften}{University of Applied Sciences Mannheim}

\newcommand{\hsmaautorbib}{\hsmaautornname, \hsmaautorvname} % Autor Nachname, Vorname
\newcommand{\hsmaautor}{\hsmaautorvname \ \hsmaautornname} % Autor Vorname Nachname

\ifthenelse{\equal{\hsmasprache}{de}}%
  {\newcommand{\hsmatyp}{\hsmatypde}%
   \newcommand{\hsmathesistype}{zur Erlangung des akademischen Grades \hsmagrad}%
   \newcommand{\hsmakoerperschaft}{\hsmakoerperschaftde}%
   \newcommand{\hsmastudiengangname}{Studiengang \hsmastudienganglangde}%
   \newcommand{\hsmastudienganglang}{\hsmastudienganglangde}%
   \newcommand{\hsmatitel}{\hsmatitelde}%
   \newcommand{\hsmatutor}{Betreuer}%
   \newcommand{\hsmafakultaetlang}{\hsmafakultaetlangde}%
   \selectlanguage{ngerman}}%
  {\newcommand{\hsmatyp}{\hsmatypen}%
   \newcommand{\hsmathesistype}{for the acquisition of the academic degree \hsmagrad}%
   \newcommand{\hsmakoerperschaft}{\hsmakoerperschaften}%
   \newcommand{\hsmastudiengangname}{Course of Studies: \hsmastudienganglang}%
   \newcommand{\hsmastudienganglang}{\hsmastudienganglangen}%
   \newcommand{\hsmatitel}{\hsmatitelen}%
   \newcommand{\hsmatutor}{Tutors}
   \newcommand{\hsmafakultaetlang}{\hsmafakultaetlangen}%
   \selectlanguage{english}}%


% Daten in die Standard Felder von KOMA-Script eintragen
\titlehead{\hsmatyp im \  \hsmastudienganglang}
\subject{}
\title{\hsmatitel}
\author{\hsmaauthor}
\date{\small{\hsmadatum}}

% Daten für das fertige PDF-Dokument
\hypersetup{
  pdftitle={\hsmatitel},  % Titel des Dokuments
  pdfauthor={\hsmaautor},              % Autor
  pdfsubject={\hsmatyp im \ \hsmastudienganglang},                % Thema
  pdfkeywords={\hsmatitel}         % Schlüsselworte
}

\newlength{\bindekorrektur}
\newlength{\seitenanfang}
\newlength{\seitenbreite}
  
\setlength{\bindekorrektur}{-46mm}   % Korrektur der horizontalen Position
\setlength{\seitenanfang}{0mm}       % Korrektur der vertikalen Position
\setlength{\seitenbreite}{297mm}

\noindent\includegraphics[width=7cm]{hsma-logo.pdf}\\

% Titel der Arbeit
\begin{textblock*}{128mm}(45mm,\seitenanfang + 62mm) % 4,5cm vom linken Rand und 6,0cm vom oberen Rand
  \centering\Large\sffamily
  \vspace{4mm} % Kleiner zusätzlicher Abstand oben für bessere Optik
  \textbf{\hsmatitel}
\end{textblock*}%

% Name
\begin{textblock*}{128mm}(45mm,\seitenanfang + 103mm)
  \centering\large\sffamily
  \hsmaautor
\end{textblock*}

% Thesis
\begin{textblock*}{\seitenbreite}(\bindekorrektur,\seitenanfang + 130mm)
  \centering\large\sffamily
  \hsmatyp\\
  \begin{small}\hsmathesistype \end{small}\\
  \vspace{2mm}
  \hsmastudiengangname
\end{textblock*}

% Fakultät
\begin{textblock*}{\seitenbreite}(\bindekorrektur,\seitenanfang + 165mm)
  \centering\large\sffamily
  \hsmafakultaetlang\\
  \vspace{2mm}
  \hsmakoerperschaft
\end{textblock*}

% Datum
\begin{textblock*}{\seitenbreite}(\bindekorrektur,\seitenanfang + 190mm)
  \centering\large 
  \textsf{\hsmadatum}
\end{textblock*}

% Firma
\begin{textblock*}{\seitenbreite}(\bindekorrektur,\seitenanfang + 215mm)
  \centering\large 
  %\textsf{Durchgeführt bei der Firma \hsmafirma}
\end{textblock*}

% Betreuer
\begin{textblock*}{\seitenbreite}(\bindekorrektur,\seitenanfang + 240mm)
  \centering\large\sffamily
  \hsmatutor \\
  \vspace{2mm}
  \hsmabetreuer\\
  \vspace{2mm}
  \hsmazweitkorrektor
\end{textblock*}

% Bibliographische Informationen
\null\newpage
\thispagestyle{empty}
  
\newcommand{\hsmabibde}{\begin{small}\textbf{\hsmaautorbib}: \\ \hsmatitelde \ / \hsmaautor. \ -- \\ \hsmatyp, \hsmaort \ : \hsmakoerperschaftde, \hsmajahr. \pageref{lastpage} Seiten.\end{small}}

\newcommand{\hsmabiben}{\begin{small}\textbf{\hsmaautorbib}: \\ \hsmatitelen \ / \hsmaautor. \ -- \\ \hsmatyp, \hsmaort \ : \hsmakoerperschaften, \hsmajahr. \pageref{lastpage} pages. \end{small}}

\ifthenelse{\equal{\hsmasprache}{de}}%
  {\hsmabibde \\ \vspace{0.5cm} \\ \hsmabiben}
  {\hsmabiben \\ \vspace{0.5cm} \\ \hsmabibde}


% Erklärung
\clearpage\setcounter{page}{1}
\thispagestyle{empty}
\textsf{\large\textbf{Erklärung}}

Hiermit erkläre ich, dass ich die vorliegende Arbeit selbstständig verfasst und keine anderen als die angegebenen Quellen und Hilfsmittel benutzt habe.

\vspace{1cm}
\hsmaort, \hsmadatum \\

\vspace{1.2cm}						                                      
\hsmaautor
\cleardoublepage

% Abstract
\chapter*{Abstract}

\ifthenelse{\equal{\hsmasprache}{de}}%
  {\subsubsection*{\hsmatitelde}\hsmaabstractde\subsubsection*{\hsmatitelen}\hsmaabstracten}
  {\subsubsection*{\hsmatitelen}\hsmaabstracten\subsubsection*{\hsmatitelde}\hsmaabstractde}


% Inhaltsverzeichnis erzeugen
\cleardoublepage
\pdfbookmark{\contentsname}{Contents}
\tableofcontents

% Korrigiert Nummerierung bei mehrseitigem Inhaltsverzeichnis
\cleardoublepage
\newcounter{frontmatterpage}
\setcounter{frontmatterpage}{\value{page}}

% Arabische Zahlen für den Hauptteil
\mainmatter

% Den Hauptteil mit vergrößertem Zeilenabstand setzen
\onehalfspacing

% ------------------------------------------------------------------
% Hauptteil der Arbeit
\chapter{Einleitung}

\section{Erster Abschnitt}

Einleitung zur Arbeit.

Möglicherweise noch einmal unterteilt in Unterabschnitte.


\subsection{Textauszeichnungen}
\label{Einleitung:Textauszeichnungen}
\index{Auszeichnungen!im Text}

Man kann Text auch \textit{kursiv} oder \textbf{fett} setzen. Es gibt Bindestrichte -, Gedankenstriche -- und lange Striche ---.


\subsection{Anführungszeichen}

Deutsche Anführungszeichen gehen so: "`dieser Text steht in \glq Anführungszeichen\grq; alles klar?"'.


\subsection{Abkürzungen}
\index{Abkürzungen}
\index{Abbreviation|see{Abkürzungen}}

Eine \ac{AKÜ} wird bei der ersten Verwendung ausgeschrieben\footnote{Ausschreiben bedeutet, dass man nicht die Abkürzung sondern die lange Form verwendet.}. Danach nicht mehr: \ac{AKÜ}. Man kann allerdings die Langform\footnote{\blindtext} explizit anfordern: \acl{AKÜ} oder die Kurzform \acs{AKÜ} oder auch noch einmal die Definition: \acf{AKÜ}.

Mehr dazu findet sich im Kapitel~\ref{Einleitung:Textauszeichnungen} auf Seite~\pageref{Einleitung:Textauszeichnungen}.




\subsubsection{Noch ein Unterabschnitt}

\paragraph{Eine Absatzüberschrift}

\blindtext[1]


\subsection{Literaturarbeit}

Wichtig ist das korrekte Zitieren von Quellen, wie es auch von \cite{Kornmeier2011} dargelegt wird. Interessant ist in diesem Zusammenhang auch der Artikel von \cite{Vixie2007}.

\blindtext[4] % Externe Datei einbinden
\chapter{Schreibstil und Typographie}


\section{Hervorhebungen}
\label{Einleitung:Textauszeichnungen}

Achten Sie bitte auf die grundlegenden Regeln der Typographie\index{Typographie}\footnote{Ein Ratgeber in allen Detailfragen ist \cite{Forssman2002}.}, wenn Sie Ihren Text schreiben. Hierzu gehören z.\,B. die Verwendung der richtigen "`Anführungszeichen"' und der Unterschied zwischen Binde- (-), Gedankenstrich (--) und langem Strich (---).

Wenn Sie Text hervorheben wollen, dann setzten Sie ihn \textit{kursiv} (Italic) und nicht \textbf{fett} (Bold). Fettdruck ist Überschriften vorbehalten; im Fließtext stört er den Lesefluss. Das \underline{Unterstreichen} von Fließtext ist im gesamten Dokument tabu und kann maximal bei Pseudo-Code vorkommen.\index{Hervorhebungen}


\section{Anführungszeichen}

Deutsche Anführungszeichen gehen so: "`dieser Text steht in \glq Anführungszeichen\grq; alles klar?"'. Englische Anführungszeichen werden anders benutzt: ``this is an `English' quotation.''


\section{Abkürzungen}
\index{Abkürzungen}
\index{Abbreviation|see{Abkürzungen}}

Eine \ac{ABK} wird bei der ersten Verwendung ausgeschrieben. Danach nicht mehr: \ac{ABK}. Man kann allerdings die Langform explizit anfordern: \acl{ABK} oder die Kurzform \acs{ABK} oder auch noch einmal die Definition: \acf{ABK}.

Beachten Sie, dass bei Abkürzungen, die für zwei Wörter stehen, ein kleines Leerzeichen nach dem Punkt kommt: z.\,B. bzw. \zb, d.\,h. bzw. \dahe.


\section{Querverweise}

Querverweise auf eine Kapitelnummer macht man im Text mit \verb+\ref+ (Kapitel~\ref{Einleitung:Textauszeichnungen}) und auf eine bestimmte Seite mit \verb+\pageref+ (Seite~\pageref{Einleitung:Textauszeichnungen}).


\section{Fußnoten}

Fußnoten werden einfach mit in den Text geschrieben und zwar genau an die Stelle\footnote{An der die Fußnote auftauchen soll.}



\section{Fremdsprachige Begriffe}

Wenn Sie Ihre Arbeit auf Deutsch verfassen, gehen Sie sparsam mit englischen Ausdrücken um. Natürlich brauchen Sie etablierte englische Fachbegriffe, wie z.\,B. \textit{Interrupt}, nicht zu übersetzen. Sie sollten aber immer dann, wenn es einen gleichwertigen deutschen Begriff gibt, diesem den Vorrang geben. Den englischen Begriff (\textit{term}) können Sie dann in Klammern oder in einer Fußnote\footnote{Englisch: \textit{footnote}.} erwähnen. Absolut unakzeptabel sind deutsch gebeugte englische Wörter oder Kompositionen aus deutschen und englischen Wörtern wie z.\,B. downgeloadet, upgedated, Keydruck oder Beautyzentrum. 



\section{Tabellen}

Tabellen werden normalerweise ohne vertikale Striche gesetzt, sondern die Spalten werden durch einen entsprechenden Abstand voneinander getrennt.\footnote{Siehe \cite[S. 89]{Willberg1999}.} Zum Einsatz kommen ausschließlich horizontale Linien (siehe Tabelle~\ref{Kap2:Kopplungsformen}).

\begin{table}[h]
  \caption{Ebenen der Kopplung und Beispiele für enge und lose Kopplung}
  \label{Kap2:Kopplungsformen}
  \renewcommand{\arraystretch}{1.2}
  \centering
  \sffamily
  \begin{footnotesize}
    \begin{tabular}{l l l}
    \toprule
    \textbf{Form der Kopplung} & \textbf{enge Kopplung} & \textbf{lose Kopplung}\\
    \midrule
    Physikalische Verbindung	&	Punkt-zu-Punkt	& 	über Vermittler\\
    Kommunikationsstil	&	synchron		&	asynchron\\
    Datenmodell	&	komplexe gemeinsame Typen	&	nur einfache gemeinsame Typen\\
    Bindung	&	statisch		&	dynamisch\\
    \bottomrule
    \end{tabular}
  \end{footnotesize}
  \rmfamily
\end{table}

Eine Tabelle fließt genauso, wie auch Bilder durch den Text. Siehe Tabelle~\ref{Kap2:Kopplungsformen}.


\section{Aufzählungen}

Aufzählungen sind toll.

\begin{itemize}
  \item Ein wichtiger Punkt
  \item Noch ein wichtiger Punkt
  \item Ein Punkt mit Unterpunkten
    \begin{itemize}
      \item Unterpunkt 1
      \item Unterpunkt 2      
    \end{itemize}
  \item Ein abschließender Punkt ohne Unterpunkte
\end{itemize}


Aufzählungen mit laufenden Nummern sind auch toll.

\begin{enumerate}
  \item Ein wichtiger Punkt
  \item Noch ein wichtiger Punkt
  \item Ein Punkt mit Unterpunkten
    \begin{enumerate}
      \item Unterpunkt 1
      \item Unterpunkt 2      
    \end{enumerate}
  \item Ein abschließender Punkt ohne Unterpunkte
\end{enumerate}


\section{Zitate}

\subsection{Zitate im Text}

Wichtig ist das korrekte Zitieren von Quellen, wie es auch von \cite{Kornmeier2011} dargelegt wird. Interessant ist in diesem Zusammenhang auch der Artikel von \cite{Kramer2009}. Häufig werden die Zitate auch in Klammern gesetzt, wie bei \citep{Kornmeier2011} und mit Seitenzahlen versehen \citep[S. 22--24]{Kornmeier2011}.


\subsection{Zitierstile}

Verwenden Sie eine einheitliche und im gesamten Dokument konsequent durchgehaltene Zitierweise\index{Zitierweise}. Es gibt eine ganze Reihe von unterschiedlichen Standards für das Zitieren und den Aufbau eines Literaturverzeichnisses. Sie können entweder mit Fußnoten oder Kurzbelegen im Text arbeiten. Welches Verfahren Sie einsetzen ist Ihnen überlassen, nur müssen Sie es konsequent durchhalten.

In der Informatik ist das Zitieren mit Kurzbelegen\index{Zitat!Kurzbeleg} im Text (Harvard-Zitierweise) weit verbreitet, wobei für das Literaturverzeichnis häufig die Regeln der \acs{ACM} oder \acs{IEEE} angewandt werden.\footnote{Einen Überblick über viele verschiedene Zitierweisen finden Sie in der \url{http://amath.colorado.edu/documentation/LaTeX/reference/faq/bibstyles.pdf}}

Denken Sie daran, dass das Übernehmen einer fremden Textstelle ohne entsprechenden Hinweis auf die Herkunft in wissenschaftlichen Arbeiten nicht akzeptabel ist und dazu führen kann, dass die Arbeit nicht anerkannt wird. Plagiate\index{Plagiat!Bewertung} werden mit mangelhaft (5,0) bewertet und können weitere rechtliche Schritte nach sich ziehen.


\subsection{Zitieren von Internetquellen}

Internetquellen\index{Zitat!Internetquellen} sind normalerweise \textit{nicht} zitierfähig. Zum einen, weil sie nicht dauerhaft zur Verfügung stehen und damit für den Leser möglicherweise nicht beschaffbar sind und zum anderen, weil häufig der wissenschaftliche Anspruch fehlt.\footnote{Eine lesenswerte Abhandlung zu diesem Thema findet sich (im Internet) bei \cite{Weber2006}}

Wenn ausnahmsweise doch eine Internetquelle zitiert werden muss, z.\,B. weil für eine Arbeit dort Informationen zu einem beschriebenen Unternehmen abgerufen wurden, sind folgende Punkte zu beachten:

\begin{itemize}
\item Die Webseite ist auszudrucken und im Anhang der Arbeit beizufügen,
\item das Datum des Abrufs und die URL sind anzugeben,
\item verwenden Sie Internet-Seiten ausschließlich zu illustrativen Zwecken (z.\,B. um einen Sachverhalt noch etwas genauer zu erläutern), aber nicht zur Faktenvermittlung (z.\,B. um eine Ihrer Thesen zu belegen).
\end{itemize}

Wenn Sie aufgrund der Natur Ihrer Arbeit sehr viele Internetquellen benötigen, dann können Sie diese statt sie auszudrucken auch in elektronischer Form abgeben (CD/DVD). Als Abgabeformat der elektronischen Quellen ist PDF/A\footnote{Bei PDF/A handelt es sich um ein besonders stabile Variante des \ac{PDF}, die von der  \ac{ISO} standardisiert wurde.} vorteilhaft, weil es von allen Formaten die größte Stabilität besitzt.
Auf der CD/DVD geben Sie bitte auch eine HTML-Version des Literaturverzeichnisses ab, in der die Online-Quellen sowie die gespeicherten PDF-Dateien verlinkt sind.

Wikipedia\index{Zitat!Wikipedia} stellt einen immensen Wissensfundus dar und enthält zu vielen Themen hervorragende Artikel. Sie müssen sich aber darüber im Klaren sein, dass die Artikel in Wikipedia einem ständigen Wandel unterworfen sind und nicht als Quelle für wissenschaftliche Fakten genutzt werden sollten. Es gelten die allgemeinen Regeln für das Zitieren von Internetquellen. Sollten Sie doch Wikipedia nutzen müssen, verwenden Sie bitte ausschließlich den Perma-Link\footnote{Sie erhalten den Permalink über die Historie der Seite und einen Klick auf das Datum.}\index{Permalink} zu der Version der Seite, die Sie aufgerufen haben.


 % Externe Datei einbinden
\chapter{Einbinden von Grafiken, Sourcecode und Anforderungen}
\label{Kap3}

\section{Bilder}

Natürlich können auch Grafiken und Bilder eingebunden werden, siehe z.\,B. Abbildung~\ref{Kap2:NasaRover}.

\begin{figure}[ht]
  \centering
  \includegraphics[width=6cm]{kapitel3/nasa_rover}
  \caption{Ein Nasa Rover}
  \label{Kap2:NasaRover}
\end{figure}

Man kann sich auch selbst ein Makro für das Einfügen von Bildern schreiben:

\bild{kapitel3/modell_point_to_point}{6cm}{Point to Point}

\begin{sidewaysfigure}
 \includegraphics[width=22cm]{kapitel3/ws-wsdl20-fehler}
  \caption{Sehr große Grafiken kann man drehen, damit sie auf die Seite passen}
  \label{Kap2:wsdl-fehler}
\end{sidewaysfigure}

\clearpage % Alle Bilder, die bisher kamen ausgeben


\section{Formelsatz}

Eine Formel gefällig? Mitten im Text $a_2 = \sqrt{x^3}$ oder als eigener Absatz (siehe Formel~\ref{Formel}):

\begin{equation}
\begin{bmatrix}
   1 &  4 &  2 \\
   4 &  0 & -3
\end{bmatrix}
        \cdot
\begin{bmatrix}
   1 &  1 &  0 \\
  -2 &  3 &  5 \\
   0 &  1 &  4
\end{bmatrix}
       {=}
\begin{bmatrix}
  -7 &  15 &  28 \\
   4 &   1 & -12
\end{bmatrix}
\label{Formel}
\end{equation}


\section{Sourcecode}

Man kann mit Latex auch ganz toll Sourcecode in den Text aufnehmen.

\subsection{Aus einer Datei}

\lstinputlisting[firstline=2,language=Java,caption={Crypter-Interface},label=lst:CrypterInterface]{\srcloc/Crypter.java}


\subsection{Inline}

\begin{lstlisting}[language=Java,caption=Methode checkKey()]
    /**
     * Testet den Schlüssel auf Korrektheit: Er muss mindestens die Länge 1
     * haben und darf nur Zeichen von A-Z enthalten.
     *
     * @param key zu testender Schlüssel
     * @throws CrypterException wenn der Schlüssel nicht OK ist.
     */
    protected void checkKey(Key key) throws CrypterException {

        // Passt die Länge?
        if (key.getKey().length == 0) {
            throw new CrypterException("Der Schlüssel muss mindestens " +
                    "ein Zeichen lang sein");
        }

        checkCharacters(key.getKey(), ALPHABET);
    }
\end{lstlisting}


\section{Anforderungen}

Anforderungen im Format des Volere"=Templates (Snowcards) \autocite{Volere} können per Makro eingefügt werden. Das Label wird automatisch mit der Nummer erstellt, d.\,h. Sie können auf die Tabelle mit dieser referenzieren (siehe \autoref{F52}).

\snowcard % Snowcard einbinden (Anpassungen in titelblatt.tex)
   {F52} % Nummer des Requirements
   {F} % Art
   {Hoch} % Priorität
   {User Authentifizierung} % Titel
   {Interview mit Abteilungsleiter} % Herkunft (Optional)
   {F12} % Konflikte (Optional)
   {Der Benutzer ist in der Lage sich über seinen
    Benutzernamen und sein Passwort am System anzumelden} % Beschreibung
   {Ein Benutzer kann sich mit seinem firmenweiten Benutzernamen und
   Passwort über die Anmeldemaske anmelden und hat Zugriff auf die
   Funktionen des Systems} % Fit-Kriterium (Optional)
   {Benutzerhandbuch des Altsystems} % Material (Optional)

Ebenso können Sie nicht"=funktionale Anforderungen mit Hilfe von Quality Attribute Scenarios (vgl. \autoref{NF11}) darstellen. Zu Details siehe \autocite{Barbacci2003}.

\qas % Quality-Attribute Scenario einbinden (Anpassungen in titelblatt.tex)
   {NF11} % Nummer des Requirements
   {Hoch} % Priotität
   {Performance des Jahresabschlusses} % Titel
   {Endbenutzer} % Quelle
   {Startet einen Jahresabschluss} % Stimulus
   {Buchhaltungssystem} % Artefakt
   {Das System befindet sich im normalen Betriebszustand} % Umgebung
   {Jahresabschluss ist durchgeführt und kann als PDF abgerufen werden} % Antwort
   {10 Minuten} % Antwort-Maß

Die Abgrenzung von funktionalen und nicht-funktionalen Anforderungen ist nicht immer einfach und bereitet manchen Studierenden Probleme. Als Hilfestellung kann die von der ISO25010 \autocite{ISO25010} zur Verfügung gestellte Liste dienen, siehe \autoref{kapitel3/iso25010}.

\bild{kapitel3/iso25010}{14cm}{Qualitätsmodell für Software-Produkte nach ISO25010}

\citeauthor{Bass2003} listen in \autocite{Bass2003} eine ähnliche Liste von Kategorien für nicht-funktionalen Anforderungen auf, die ebenfalls als Richtschnur dienen kann. Diese sind:

\begin{itemize}
  \item \textit{Verfügbarkeit} \textit{(availability)} -- umfasst Zuverlässigkeit (reliability), Robustheit (robustness), Fehlertoleranz (fault tolerance) und Skalierbarkeit (scalability)
  \item \textit{Anpassbarkeit} \textit{(modifiability)}, umfasst Wartbarkeit (maintainability), Verständlichkeit (understandability) und Portabilität (portability).
  \item \textit{Performanz} \textit{(performance)}
  \item \textit{Sicherheit} \textit{(security)}
  \item \textit{Testbarkeit} \textit{(testability)}
  \item \textit{Bedienbarkeit} \textit{(usability)}
\end{itemize}
 % Externe Datei einbinden
% ------------------------------------------------------------------

\label{lastpage}

% Neue Seite
\cleardoublepage

% Backmatter mit normalem Zeilenabstand setzen
\singlespacing

% Römische Ziffern für die "Back-Matter", fortlaufend mit "Front-Matter"
\pagenumbering{roman}
\setcounter{page}{\value{frontmatterpage}}

% Abkürzungsverzeichnis
\addchap{\hsmaabbreviations}
\chapter*{Abkürzungsverzeichnis}
\addcontentsline{toc}{chapter}{Abkürzungsverzeichnis}

\begin{acronym}
\acro{ABK}{Abkürzung}
\acro{ACM}{Association of Computing Machinery}
\acro{PDF}{Portable Document Format}
\acro{IEEE}{Institute of Electrical and Electronics Engineers}
\acro{ISO}{International Organization for Standardization}
\end{acronym}


% Tabellenverzeichnis erzeugen
\cleardoublepage
\phantomsection
\addcontentsline{toc}{chapter}{\hsmalistoftables}
\listoftables

% Abbildungsverzeichnis erzeugen
\cleardoublepage
\phantomsection
\addcontentsline{toc}{chapter}{\hsmalistoffigures}
\listoffigures

% Listingverzeichnis erzeugen. Wenn Sie keine Listings haben,
% entfernen Sie einfach diesen Teil.
\cleardoublepage
\phantomsection
\addcontentsline{toc}{chapter}{\hsmalistings}
\lstlistoflistings

% Literaturverzeichnis erzeugen
\begingroup
\cleardoublepage
\begin{flushleft}
\let\clearpage\relax % Fix für leere Seiten (issue #25)
\printbibliography
\end{flushleft}
\endgroup

% Index ausgeben. Wenn Sie keinen Index haben, entfernen Sie einfach
% diesen Teil.
\cleardoublepage
\phantomsection
\addcontentsline{toc}{chapter}{\hsmaindex}
\printindex

% Anhang. Wenn Sie keinen Anhang haben, entfernen Sie einfach
% diesen Teil.
\appendix
\chapter{Erster Anhang}

Hier ein Beispiel für einen Anhang. Der Anhang kann genauso in Kapitel und Unterkapitel unterteilt werden, wie die anderen Teile der Arbeit auch.

\chapter{Zweiter Anhang}

Hier noch ein Beispiel für einen Anhang.


\end{document}
